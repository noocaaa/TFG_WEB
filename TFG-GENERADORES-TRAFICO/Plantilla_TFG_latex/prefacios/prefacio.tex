\chapter*{}
\thispagestyle{empty}
\begin{center}

\large\bfseries Sistema de tutorización inteligente para la enseñanza de programación a niños\\[1cm]

\end{center}

\begin{center}

Noelia Carrasco Vilar \\

\end{center}

\vspace{0.5cm}
\noindent{\textbf{Palabras clave}:
\textit{Sistema de Tutorización Inteligente},
\textit{Educación Infantil},
\textit{Análisis},
\textit{Automatización},
\textit{Programación}}
\vspace{0.5cm}

\noindent{\textbf{Resumen}}\\

La enseñanza de la programación a edades tempranas se ha vuelto fundamental en la sociedad actual. Es en este contexto que los Sistemas de Tutorización Inteligente (ITS) emergen como herramientas potenciales para apoyar un aprendizaje más personalizado y adaptado a las necesidades individuales de cada estudiante.

Este trabajo presenta el diseño, desarrollo y evaluación de un ITS enfocado en la enseñanza de programación a niños. El sistema propuesto aplica técnicas de automatización en la corrección de los ejercicios con tal de ofrecer una buena retroalimentación del entendimiento del estudiante en tiempo real. Asimismo, es capaz de adaptarse a las necesidades individuales de los estudiantes y monitorizar su progreso. 

El proceso de desarrollo del ITS se llevó a cabo en diversas fases. En primer lugar, se establecieron los fundamentos pedagógicos y se identificaron las necesidades específicas del público infantil. Posteriormente, se integraron técnicas avanzadas de corrección y elección de ejercicios con tal de garantizar la adaptabilidad y efectividad del sistema.

Para evaluar la eficacia del ITS, se realizaron pruebas y experimentos con un grupo de tres estudiantes, en colaboración con la empresa CODELEARN S.L. Los resultados iniciales sugieren una posible mejora significativa en la retención de conocimientos y en el rendimiento académico de los participantes, validando así la propuesta de este TFG.

En conclusión, este Trabajo de Fin de Grado aporta una solución innovadora al campo de la educación tecnológica infantil, demostrando la viabilidad y eficacia de los Sistemas de Tutorización Inteligente dentro del mundo educativo de la programación.


% ABSTRACT

\chapter*{}
\thispagestyle{empty}

\begin{center}

{\large\bfseries "Intelligent Tutoring System for Teaching Programming to Children}\\[1cm]

\end{center}

\begin{center}

Noelia Carrasco Vilar\\

\end{center}

\vspace{0.5cm}
\noindent{\textbf{Keywords}:
\textit{Intelligent Tutoring System},
\textit{Early Childhood Education},
\textit{Analysis},
\textit{Automation},
\textit{Programming}}
\vspace{0.5cm}

\noindent{\textbf{Abstract}}\\

Teaching programming at an early age has become essential in today's society. It is in this context that Intelligent Tutoring Systems (ITS) emerge as potential tools to support more personalized learning tailored to the individual needs of each student.

This work presents the design, development, and evaluation of an ITS focused on teaching programming to children. The proposed system applies automation techniques in exercise correction to provide real-time feedback on the student's understanding. It is also capable of adapting to individual student needs and monitoring their progress.

The development of the ITS was carried out in various phases. Firstly, pedagogical foundations were established, and the specific needs of the child audience were identified. Advanced correction and exercise selection techniques were then integrated to ensure the adaptability and effectiveness of the system.

To evaluate the effectiveness of the ITS, tests and experiments were conducted with a group of three students, in collaboration with the company CODELEARN S.L. Initial results suggest a possible significant improvement in knowledge retention and academic performance of the participants, thus validating the proposal of this Bachelor's Thesis.

In conclusion, this Bachelor's Thesis provides an innovative solution to the field of early childhood technological education, demonstrating the feasibility and effectiveness of Intelligent Tutoring Systems in the educational programming world.

\chapter*{}
\thispagestyle{empty}

\noindent\rule[-1ex]{\textwidth}{2pt}\\[4.5ex]

Yo, \textbf{Noelia Carrasco Vilar}, alumna de la titulación Grado en Ingeniería Informática de la \textbf{Escuela Técnica Superior
de Ingenierías Informática y de Telecomunicación de la Universidad de
Granada}, con DNI 39475493H, autorizo la
ubicación de la siguiente copia de mi Trabajo Fin de Grado en la biblioteca
del centro para que pueda ser
consultada por las personas que lo deseen.

\vspace{6cm}

\noindent Fdo: Noelia Carrasco Vilar

\vspace{2cm}

\begin{flushright}
Granada a 13 de Noviembre de 2023.
\end{flushright}


\chapter*{}
\thispagestyle{empty}

\noindent\rule[-1ex]{\textwidth}{2pt}\\[4.5ex]

D. \textbf{José Manuel Benítez Sánchez}, catédratico del departamento de Ciencias de la Computación e Inteligencia Artificial de la Universidad de Granada.


\vspace{0.5cm}

\textbf{Informa:}

\vspace{0.5cm}

Que el presente trabajo, titulado \textit{\textbf{ Sistema de tutorización inteligente para la enseñanza de programación a niños}}, ha sido realizado bajo su supervisión por \textbf{Noelia Carrasco Vilar}, y autoriza la defensa de dicho trabajo ante el tribunal que corresponda.

\vspace{0.5cm}

Y para que conste, expide y firma el presente informe en Granada a 13 de
Noviembre de 2023.

\vspace{1cm}

\textbf{El director:}

\vspace{5cm}

\noindent \textbf{José Manuel Benítez Sánchezz}

\chapter*{Agradecimientos}

Quiero comenzar agradeciendo a loz profesores de la carrera que han realizado un compromiso para la formación académica de mi promoción. Su pasión y dedicación fomentaron mi interés por la informática, lo cual me ha llevado a realizar este trabajo de fin de grado. Un agradecimiento especial a mi tutor del TFG, quien además de guiar este proyecto, avivó mi interés por la investigación en los Sistemas de Tutorización Inteligente.

Agradezco de manera especial a la empresa Codelearn S.L. por su apoyo y colaboración en este proyecto. Su experiencia y recursos han sido cruciales para el desarrollo de mi TFG.

No puedo pasar por alto el papel fundamental que mi familia ha desempeñado en mi vida, tanto académica como personal. A mis padres y a mi hermano, cuya influencia ha sido determinante en mi formación, dedico este trabajo y los proyectos futuros, como reconocimiento a todos los esfuerzos que han realizado para permitirme alcanzar mis metas académicas.  También deseo honrar la memoria de mis abuelos, quienes, presentes o ausentes, han sido una fuente de inspiración constante a lo largo de mi trayectoria.


