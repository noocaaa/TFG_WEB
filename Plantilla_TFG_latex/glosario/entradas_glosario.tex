\section*{Glosario} \label{glosario}

\begin{description}
    \item[ITS:] Sistema Inteligente de Tutorización (\textit{Intelligent Tutoring System}). Es un sistema que utiliza tecnologías de inteligencia artificial para proporcionar enseñanza personalizada y apoyo educativo a los estudiantes, adaptándose a las necesidades individuales de aprendizaje.
    \item[\textit{Hash} y \textit{Salting}:] Técnicas de seguridad para el almacenamiento seguro de contraseñas. El \textit{hasing} convierte una contraseña en una cadena de caracteres única, mientras que \textit{salting} añade datos aleatorios al \textit{hash} para aumentar la seguridad.
    \item[HTTPS:] Protocolo seguro para la transferencia de datos en la web. Utiliza el protocolo SSL/TLS para encriptar la información transmitida.
    \item[Flujo:] Secuencia de pasos o actividades en un proceso o escenario. En el contexto de programación, se refiere a la ejecución ordenada de instrucciones en un programa.
    \item[\textit{Web Responsive}:] Diseño web adaptable a diferentes dispositivos. Utiliza hojas de estilo en cascada (CSS) para ajustar el diseño según el tamaño de la pantalla.
    \item[Gamificación:] Uso de elementos de juego en entornos no lúdicos. Se emplea para aumentar la motivación y la participación en actividades que normalmente se considerarían tediosas.
    \item[Aprendizaje Basado en Problemas:] Método pedagógico que utiliza problemas reales o simulados como punto de partida para el aprendizaje.
    \item[\textit{API RESTful}:] Interfaz de programación de aplicaciones que sigue los principios \textit{REST} para la comunicación entre sistemas.
    \item[Backend:] Parte del sistema que se encarga de la lógica de negocio, el acceso a la base de datos y la comunicación con diferentes servicios.
    \item[Clave Primaria:] Campo único en una tabla de base de datos que se utiliza para identificar registros de forma única.
    \item[Clave Foránea:] Campo en una tabla de base de datos que se utiliza para establecer una relación con otra tabla.
    \item[Normalización:] Proceso de organización de una base de datos para reducir la redundancia y mejorar la integridad de los datos.
    \item[Análisis de Árbol de Sintaxis Abstracta:] Técnica de análisis de código que genera un árbol para representar su estructura sintáctica.
    \item[Complejidad Ciclomática:] Métrica que mide la complejidad de un programa basada en el número de caminos independientes a través del código.
    \item[Función Indicadora:] Función matemática que toma el valor de 1 si una condición específica se cumple y 0 en caso contrario.
    \item[Requisitos de Aprendizaje:] Objetivos educativos específicos que un estudiante debe alcanzar en un sistema de aprendizaje adaptativo.
    \item[Tasa de Fracaso:] Proporción de ejercicios o tareas que un estudiante no ha podido completar con éxito.
    \item[Ponderación:] Asignación de importancia relativa a diferentes criterios o variables en un cálculo.
    \item[Análisis Estático:] Búsqueda en el código de problemas sin ejecutar el programa.
    \item[Adaptabilidad:] Capacidad de un sistema para ajustarse dinámicamente a las necesidades o comportamientos del usuario.
    \item[Fitness:] Métrica que evalúa la calidad o eficacia de una solución en comparación con una solución de referencia.
    \item[Indicadores Cuantitativos:] Métricas numéricas utilizadas para evaluar el rendimiento o eficacia de un sistema.
    \item[Usabilidad:] Medida de la eficacia, eficiencia y satisfacción con la que los usuarios pueden realizar tareas en un sistema.
    \item[Metodología de Pruebas Individuales:] Enfoque de evaluación que implica probar el sistema con cada usuario de forma separada para obtener retroalimentación más precisa.
    \item[Optimización:] Proceso de cambios en un sistema para mejorarlo en términos de eficiencia, eficacia o experiencia del usuario.
\end{description}
