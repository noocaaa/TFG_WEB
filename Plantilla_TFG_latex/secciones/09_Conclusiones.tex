\chapter{Conclusiones y Trabajo Futuro} \label{chap:conclusiones}

Este último capítulo sirve como el cierre al Trabajo de Fin de Grado presentado. En él, se busca sintetizar los hallazgos más significativos, destacando cómo estos contribuyen al campo de la enseñanza de la programación a niños. Asimismo, se esbozarán las direcciones para el trabajo futuro, delineando oportunidades para la expansión. Finalmente, se ofrecerá una reflexión personal sobre el impacto del proyecto.

\section{Conclusiones}

Este TFG se ha centrado en un reto importante: la enseñanza de la programación a niños en el contexto de una sociedad digitalizada en rápida evolución. A través de un enfoque multidisciplinario \ref{chap:state_of_the_art}, combinando pedagogía, tecnología educativa y diseño de sistemas, se ha desarrollado una plataforma innovadora y adaptable, destacándose en el campo de la enseñanza.

Un logro clave ha sido el desarrollo del Sistema de Tutorización Inteligente, que supera las barreras de los métodos tradicionales al ofrecer un aprendizaje personalizado y adaptativo. Este sistema no solo mejora la calidad del aprendizaje, sino que también permite a los estudiantes progresar a su propio ritmo, manteniendo las buenas prácticas de programación.

La colaboración con Codelearn S.L. ha sido instrumental, fusionando teoría y práctica para evaluar y mejorar el ITS, y así dar los primeros pasos hacia un sistema educativo más eficaz y atractivo.

\section{Trabajo Futuro}

A pesar de los avances logrados, el camino por delante es largo. Futuras líneas de investigación y desarrollo incluyen:

\begin{itemize}
    \item \textbf{Variedad de Recursos Didácticos}: Una mayor diversidad en el tipo de ejercicios y evaluaciones, cómo relleno de código o cuestionarios sobre ciertos temas, entre otros ejemplos.
    \item \textbf{Automatización  de la Corrección}: Conseguir automatizar la corrección para los ejercicios HTML y que no se recurra al profesor. 
    \item \textbf{Sistema de Chat-Bot}: Sistema conversacional para una primera resolución de dudas comunes en un ejercicio, como una explicación más extensa o ejemplos de ejecución.  
    \item \textbf{Inteligencia Artificial}: El uso de algoritmos de aprendizaje automático que optimizaría la adaptabilidad del sistema a las necesidades individuales del estudiante, debido a la gran cantidad de datos con la que entrenaría.
    \item \textbf{Feedback en Tiempo Real}: La implementación de análisis del código mientras el estudiante lo este escribiendo con tal de poder hacer ajustes y/o comentarios inmediatos.
    \item \textbf{Tests A/B}: Realización de pruebas para evaluar la eficacia de distintas técnicas pedagógicas y elementos de diseño.
  \end{itemize}

\section{Reflexión Personal}
Desde la concepción inicial del proyecto hasta la implementación técnica y las pruebas finales, cada etapa ha contribuido a un desarrollo profesional.

Empezando con la colaboración con Codelearn S.L., la experiencia ha sido invaluable. Ver un entorno real me ha ofrecido perspectivas que el ámbito académico raramente puede proporcionar. He aprendido a abordar problemas desde un punto de vista más aplicado, equilibrando teoría y práctica. Fortaleciendo, así, mi comprensión sobre cómo llevar soluciones tecnológicas desde el tablero de dibujo hasta una implementación funcional y real.

En el aspecto académico, este proyecto ha servido como una especie de trampolín para relacionar la ingeniería de software, los algoritmos y la pedagogía. Ganando una visión más clara de cómo la mayoría de disciplinas se pueden fusionar para producir soluciones educativas efectivas. 

El desarrollo ágil adoptado para el proyecto fue una lección valiosa en sí misma. Me enseñó la importancia de la adaptabilidad y la agilidad en un entorno tecnológico que nunca se detiene. La retroalimentación continua y los ajustes constantes fueron cruciales para navegar a través de los desafíos que inevitablemente surgieron.

