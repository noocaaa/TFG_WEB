\chapter{Conclusiones y Trabajo Futuro} \label{chap:conclusiones}

Este último capítulo sirve como el cierre al Trabajo de Fin de Grado presentado. En él, se busca sintetizar los hallazgos más significativos, destacando cómo estos contribuyen al campo de la enseñanza de la programación a niños. Asimismo, se esbozarán las direcciones para el trabajo futuro, delineando oportunidades para la expansión. Finalmente, se ofrecerá una reflexión personal sobre el impacto del proyecto.

\section{Conclusiones}

El Trabajo de Fin de Grado se ha centrado en un reto importante: la enseñanza de la programación a niños. Y lo hace en el contexto de una sociedad digitalizada en rápida evolución. Su distintivo radica en un enfoque multidisciplinario \ref{chap:state_of_the_art}. Pedagogía, tecnología educativa y diseño de sistemas se unen para crear una plataforma más completa y adaptable.


Uno de los pilares fundamentales de este TFG ha sido el desarrollo del Sistema de Tutorización Inteligente. Supera las limitaciones de los métodos tradicionales, que a menudo carecen de adaptabilidad. Más aún, el ITS permite un aprendizaje personalizado y eficiente. Los alumnos aprenden a su ritmo, manteniendo una alta calidad en su código.

La colaboración con Codelearn S.L. ha añadido un valor significativo a este proyecto, permitiendo una implementación y evaluación más práctica del ITS. Esta relación ha sido fundamental para fusionar teoría y práctica, lo que resulta en los primeros pasos de un sistema de enseñanza más efectivo.

\section{Trabajo Futuro}

Este proyecto ha avanzado mucho, pero el camino por delante es largo. Futuras líneas de investigación y desarrollo incluyen:


\begin{itemize}
    \item \textbf{Variedad de Recursos Didácticos}: Una mayor diversidad en el tipo de ejercicios y evaluaciones, cómo relleno de código o cuestionarios sobre ciertos temas, entre otros ejemplos.
    \item \textbf{Automatización  de la Corrección}: Conseguir automatizar la corrección para los ejercicios HTML y no se recurra al profesor. 
    \item \textbf{Sistema de Chat-Bot}: Sistema de conversacional para una primera resolución de dudas comunes en un ejercicio, como una explicación más extensa o ejemplos de ejecución.  
    \item \textbf{Inteligencia Artificial}: El uso de algoritmos de aprendizaje automático que optimizaría la adaptabilidad del sistema a las necesidades individuales del estudiante, debido a la gran cantidad de datos con la que entrenaría.
    \item \textbf{Feedback en Tiempo Real}: La implementación de análisis del código mientras el estudiante lo este escribiendo con tal de poder hacer ajustes y/o comentarios inmediatios.
    \item \textbf{Tests A/B}: Realización de pruebas para evaluar la eficacia de distintas técnicas pedagógicas y elementos de diseño.
  \end{itemize}

\section{Reflexión Personal}

Trabajar en este TFG ha sido tanto desafiante como gratificante. La oportunidad de colaborar con una entidad como Codelearn S.L. me ha ofrecido una visión práctica invaluable, complementando mi formación académica. Este proyecto ha consolidado mi entendimiento de cómo se pueden aplicar las teorías pedagógicas en entornos tecnológicos y ha reforzado mi interés en la intersección entre tecnología y educación. Además, la metodología ágil adoptada para el desarrollo del proyecto ha subrayado la importancia de la flexibilidad y adaptabilidad en un entorno tecnológico en constante cambio.

\section{Reflexión Personal}
Desde la concepción inicial del proyecto hasta la implementación técnica y las pruebas finales, cada etapa del TFG ha contribuido a un desarrollo profesional.

Empezando con la colaboración con Codelearn S.L., la experiencia ha sido especialmente enriquecedora. Trabajar en un entorno real me ha ofrecido perspectivas que el ámbito académico raramente puede proporcionar. He aprendido a abordar problemas desde un punto de vista más aplicado, equilibrando teoría y práctica. Esto ha servido para fortalecer mi comprensión sobre cómo llevar soluciones tecnológicas desde el tablero de dibujo hasta una implementación funcional y real.

En el aspecto académico, este proyecto ha servido como una especie de trampolín para relacionar la ingeniería de software, los algoritmos y la pedagogía. He ganado una visión más clara de cómo la mayoría de disciplinas se pueden fusionar para  producir soluciones educativas efectivas. 

El desarrollo ágil adoptado para el proyecto fue una lección invaluable en sí misma. Me enseñó la importancia de la adaptabilidad y la agilidad en un entorno tecnológico que nunca se detiene. La retroalimentación continua y los ajustes constantes fueron cruciales para navegar a través de los desafíos que inevitablemente surgieron.

