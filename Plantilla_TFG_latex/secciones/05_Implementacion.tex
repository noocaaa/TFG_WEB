\chapter{Análisis} \label{chap:analisis}

\section{Análisis}

En este capítulo se van a exponer diferentes aspectos relacionados con el diseño e implementación. Concretamente, se realiza un análisis de los usuarios objetivo, comprendiendo sus necesidades, expectativas y limitaciones. Asimismo, también se muestran los modelos de caso de uso, la arquitectura del sistema. Conjuntamente, con los modelos de comportamiento y el modelado de entidades y relaciones. De esta manera, se ofrece una visión clara y estructurada de la solución propuesta.

\subsection{Análisis de Usuarios Objetivo}

El enfoque de la plataforma web está dirigido principalmente a niños entre los 8 y 18 años, siguiendo la metodología de CODELEARN S.L.. Donde la mayoría de los clientes son estudiantes con poco o ningún conocimiento previo de programación. Es por ello que la plataforma está diseñada para ser intuitiva y flexible, en línea con las recomendaciones de Large y Beheshti sobre el diseño de interfaces para niños \cite{large2005}.

Si tenemos en cuenta la brevedad del período de atención de los niños y su desarrollo cognitivo en curso, se deberán evitar tareas extensas o excesivamente complicadas. Las expectativas y necesidades son diversas, pero comúnmente incluyen una interfaz visualmente atractiva, instrucciones claras y una retroalimentación inmediata tal como especifican Duglas Frye y Elliot Soloway \cite{interfacedesign}. Asimismo, se han incorporado elementos de gamificación dentro la plataforma con tal mantener el interés y la motivación de los usuarios. 

Finalmente, la plataforma será accesible; es decir, se podrá acceder desde múltiples dispositivos como ordenadores, tablets y smartphones debido a su diseño \textit{Web Responsive}.

\subsection{Análisis de Requisitos Educativos}

La enseñanza de habilidades de programación dentro de las escuelas está cobrando una gran relevancia. Sin embargo, hay una falta de herramientas y enfoques efectivos mucho más allá de herramientas como Scratch \cite{teachingcodingchildren}. Conscientes de esta brecha, este sistema educativo adopta un enfoque fundamentado en evidencia y prácticas académicas sólidas.

Aunque lenguajes como Scratch y Logo son de los más populares para la educación de niños, se ha optado por Python, C++, Java y HTML+CSS+JS. Python es conocido por su simplicidad y legibilidad, lo que facilita el aprendizaje para los principiantes \cite{pears2007}. Además, permite una transición fluida a lenguajes más complejos. C++, Java y HTML+CSS+JS, aunque sujetos a debate sobre su idoneidad para principiantes, son cruciales en la industria y la educación. Su relevancia práctica justifica su presencia \cite{dewarschonberg}.

Al ir más allá de los enfoques basados en bloques como Scratch, se busca que los niños aprendan la programación basada en texto, más cercana a la realidad. Los ejercicios se organizan por competencias específicas, permitiendo un aprendizaje adaptable y progresivo. Alineando el enfoque con los resultados de investigaciones sobre el tema \cite{document2010}. 

Es importante destacar que el sistema se ha diseñado para ser inclusivo, ofreciendo ejercicios de diversa dificultad y con un buen sistema de asistencia. De esta manera, se siguen las directrices recomendadas por Luxton-Reily y Wünsche sobre las características de un buen sistema inteligente para la enseñanza de programación \cite{intelligentturoingprogrammingeducation}, consiguiendo un buen enfoque educativo.

\section{Modelos de Caso de Uso}

\subsection{Caso de Uso para Estudiantes}
% Contenido aquí

\subsection{Caso de Uso para Administradores}
% Contenido aquí (si aplicable)

\section{Arquitectura del Sistema}

\subsection{Arquitectura Frontend}
% Contenido aquí

\subsection{Arquitectura Backend}
% Contenido aquí

\section{Modelos de Comportamiento}

\subsection{Diagramas de Flujo para la Selección y Evaluación de Ejercicios}
% Contenido aquí

\subsection{Diagramas para la Lógica de Corrección y Evaluación de Calidad de Código}
% Contenido aquí

\section{Modelado de Entidades y sus Relaciones}

\subsection{Entidades Principales}
% Contenido aquí

\subsection{Diagramas ER o de Clases}
% Contenido aquí

\section{Métricas de Evaluación}
% Contenido aquí
