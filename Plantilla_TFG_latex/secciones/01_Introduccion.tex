\chapter{Introducción} \label{chap:introduction}

\section{Contexto Actual y Relevancia de la Programación}
En la era actual, la digitalización está en todas partes. Afecta desde cómo nos comunicamos hasta como llevamos a cabo de tareas cotidianas. Siendo la tecnología la impulsora de este cambio. Dentro de este contexto,  la programación ya no es una mera herramienta técnica. Si no, que se ha convertido en una habilidad esencial del siglo XXI. No solo es importante para aquellos que buscan una carrera tecnológica. También es importante para el público que desea entender y utilizar eficazmente las herramientas digitales. Sin embargo, enseñar programación a niños presenta desafíos únicos. Debido a la complejidad del contenido como a las necesidades pedagógicas específicas.

\section{Descripción del Problema}

Tal como hemos hablado en la sección anterior, enseñar programación a niños es un desafío que no podemos ignorar. Existen diversas plataformas educativas, pero muchas tienen bastantes limitaciones. Ofrecen ejercicios y evaluaciones estáticas, sin tener en cuenta una adaptación al ritmo o habilidades del estudiante. Además, la mayoría, únicamente se centran en la corrección del código. Por ello, ignoran aspectos como la eficiencia del mismo y el uso de buenas prácticas de programación. Este enfoque limitado no solo afecta a los estudiantes. También deja a profesores y administradores sin las herramientas necesarias para interactuar de una forma efectiva con el sistema. Es por eso que hay una necesidad clara e importante de una solución más completa y adaptativa para la enseñanza de programación a niños.

\section{Enfoque}
Este TFG adopta un enfoque multidisciplinario, combinando elementos de pedagogía, tecnología educativa y diseño de sistemas. La naturaleza del trabajo exige la interacción de varias disciplinas con tal de abordar de manera eficaz los desafíos presentados por la enseñanza de programación a niños.

\subsection{Pilares del Proyecto}
El trabajo se estructura alrededor de varios núcleos esenciales:

\begin{enumerate}
\item Análisis del papel actual de la programación en la sociedad digital.
\item Evaluación de los desafíos pedagógicos en la enseñanza de programación a niños.
\item Diseño y estructura del Sistema de Tutorización Inteligente (ITS).
\item Metodología de adaptación del ITS basada en las respuestas y progreso del estudiante.
\item Integración del ITS en una plataforma web, enfocándose en la experiencia del usuario.
\item Experimentación y análisis en contextos reales de enseñanza.
\end{enumerate}

\subsection{Desarrollo Ágil}
Para abordar la dinámica cambiante y las necesidades de adaptabilidad del proyecto, se ha optado por usar un enfoque de desarrollo ágil para el desarrollo del software. Esto, favorece un proceso iterativo e incremental, permitiendo varios ajustes rápidos ante cualquier cambio. Además, de ofrecer la posibilidad de realizar entregas parciales que se pueden evaluar y ajustar en fases iniciales.

Esta estrategia de desarrollo asegura flexibilidad, fomenta la colaboración y se centra en el usuario final. Garantizando, así, que las soluciones implementadas se alinean con las necesidades del ámbito de la enseñanza y el aprendizaje.

\section{Colaboración con Codelearn S.L.}
Este proyecto no habría sido posible sin la colaboración estrecha con Codelearn S.L.  \cite{codelearn}, una destacada entidad internacional de la educación tecnológica para niños. A pesar de los avances y logros de Codelearn S.L., se identificó la necesidad de evolucionar y adaptar su sistema pedagógico para hacerlo más personalizado y efectivo. Esta colaboración ha permitido fusionar la experiencia y conocimientos de ambas partes, con el objetivo de desarrollar un ITS que permita mejorar la forma en que se enseña programación a niños.

\section{Conclusión}
Este TFG busca ofrecer soluciones sólidas y bien argumentadas a los retos actuales de la enseñanza en la programación. Mediante un enfoque innovador y una aplicación práctica, se busca establecer un fundamento sobre el cual puedan construirse futuros avances e investigaciones dentro del campo de la tecnología educativa adaptativa.
