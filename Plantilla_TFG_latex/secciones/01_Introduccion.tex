\chapter{Introducción} \label{chap:introduction}

\section{Contexto Actual y Relevancia de la Programación}
En el entorno contemporáneo, la digitalización ha permeado casi todos los aspectos de nuestra vida cotidiana. Desde la forma en que comunicamos hasta cómo llevamos a cabo tareas básicas, todo se ha visto transformado bajo el influjo de la tecnología. En este contexto, la programación, que antes era vista únicamente como una herramienta técnica, ahora es una habilidad fundamental del siglo XXI. Esta no solo es crucial para aquellos que aspiran a una carrera en el ámbito tecnológico, sino también para la población en general que busca comprender y aprovechar al máximo las herramientas digitales actuales. Sin embargo, la enseñanza de programación a una audiencia joven, como los niños, plantea desafíos singulares debido a la complejidad del contenido y a las particularidades pedagógicas que requiere.

\section{Motivación y Objetivo del TFG}
Ante la creciente demanda de educación tecnológica de calidad para los más jóvenes, surge la necesidad de herramientas pedagógicas innovadoras que se adapten a este público en particular. Es en este contexto donde se sitúa el presente Trabajo de Fin de Grado (TFG). Con la ambición de ofrecer una solución didáctica adaptada, se ha desarrollado un Sistema de Tutorización Inteligente (ITS). Más allá de ofrecer lecciones y ejercicios convencionales, el ITS está diseñado para evaluar y adaptarse a las necesidades individuales de cada estudiante, usando indicadores específicos como la puntuación obtenida en la resolución de un ejercicio y el tiempo de respuesta.

\section{Enfoque}
Este TFG adopta un enfoque multidisciplinario, combinando elementos de pedagogía, tecnología educativa y diseño de sistemas. La naturaleza del trabajo exige la interacción de varias disciplinas con tal de abordar de manera eficaz los desafíos presentados por la enseñanza de programación a niños.

\subsection{Pilares del Proyecto}
El trabajo se estructura alrededor de varios núcleos esenciales:

\begin{enumerate}
\item Análisis del papel actual de la programación en la sociedad digital.
\item Evaluación de los desafíos pedagógicos en la enseñanza de programación a niños.
\item Diseño y estructura del Sistema de Tutorización Inteligente (ITS).
\item Metodología de adaptación del ITS basada en las respuestas y progreso del estudiante.
\item Integración del ITS en una plataforma web, enfocándose en la experiencia del usuario.
\item Experimentación y análisis en contextos reales de enseñanza.
\end{enumerate}

\subsection{Desarrollo Ágil}
Dada la naturaleza dinámica y la necesidad de adaptabilidad del proyecto, se ha optado por emplear un desarrollo ágil para el desarrollo del software. Este enfoque permite un desarrollo iterativo e incremental, facilitando adaptaciones rápidas a los cambios y proporcionando entregas parciales que pueden ser evaluadas y mejoradas en etapas tempranas del proyecto.

Esta metodología asegura que el desarrollo sea flexible, colaborativo y orientado al usuario final, garantizando que las soluciones desarrolladas estén alineadas con las necesidades reales de enseñanza y aprendizaje.

\section{Colaboración con Codelearn S.L.}
Este proyecto no habría sido posible sin la colaboración estrecha con Codelearn S.L.  \cite{codelearn}, una destacada entidad en el mundo de la educación tecnológica para niños. A pesar de los avances y logros de Codelearn S.L., se identificó la necesidad de evolucionar y adaptar su sistema pedagógico para hacerlo más personalizado y efectivo. Esta colaboración ha permitido fusionar la experiencia y conocimientos de ambas partes, con el objetivo de desarrollar un ITS que permita mejorar la forma en que se enseña programación a niños.

\section{Conclusión}
A través de este TFG, se busca proporcionar una respuesta concreta y bien fundamentada a los desafíos actuales de la enseñanza de programación. Con un enfoque innovador y una aplicación práctica, se espera que este trabajo siente las bases para futuras investigaciones y desarrollos en el ámbito de la educación tecnológica adaptativa.