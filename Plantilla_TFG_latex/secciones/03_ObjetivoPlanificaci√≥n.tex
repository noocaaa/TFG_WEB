\chapter{Planificación} \label{chap:planification}

La planificación es un aspecto importante en cualquier proyecto, debido a que permite asegurar una correcta ejecución y el logro de los objetivos establecidos. Por ello, en este capítulo, se aborda la organización y estructuración del proyecto con el fin de cumplir el objetivo principal. Además, se adjunta un diagrama de Gantt \ref{fig:gantt_diagram} para ofrecer una visión detallada del tiempo asignado a cada tarea.

\section{Fases}

\subsection{Fase Inicial}

En la fase inicial se establecieron las bases y objetivos del proyecto. Se reconoció la necesidad de una plataforma educativa digital que enseñase tanto programación como sus buenas prácticas. Se llevaron a cabo reuniones preliminares para esbozar las características deseadas y el alcance del proyecto, teniendo siempre en mente la experiencia de usuario y la eficiencia en la corrección y retroalimentación por parte del usuario.

\subsection{Fase de análisis}

Durante esta etapa se realizó una exploración profunda y meticulosa de las necesidades y los requerimientos clave para diseñar una plataforma de aprendizaje eficaz. Se buscó conseguir un entorno que proporcionará una buena retroalimentación, poniendo énfasis en un enfoque adaptativo y personalizado, teniendo en cuenta la diversidad de ritmos y estilos de aprendizaje. Sé investigaron tanto las demandas funcionales como las tecnológicas, y se realizó un análisis de la plataforma Codelearn S.L. con el fin de garantizar que las decisiones tomadas fueran adecuadas y eficientes.

Cabe destacar que antes de la construcción de la plataforma, fue esencial definir con precisión qué funcionalidades eran fundamentales. A través de reuniones, lecturas e investigación se definieron las siguientes características principales para el sistema:

\begin{itemize}
    \item Un sistema de corrección automatizada que evalúe tanto la solución como la calidad del código.
    \item Un sistema de retroalimentación que ofrezca comentarios constructivos.
    \item Un camino de ejercicios que se adapte dinámicamente al progreso y habilidades del estudiante.
    \item Un ranking para motivar a los estudiantes durante su aprendizaje.
    \item Una plataforma sencilla y visual, diseñada específicamente para niños, que les permita interactuar con el contenido de manera intuitiva.
    \item Un compilador que permita la ejecución y prueba de código en la propia web.
    \item Un sistema de soporte y tutoría basado en un mecanismo de preguntas y respuestas para asistir a los estudiantes con dudas específicas.
\end{itemize}

Cuando los requisitos y objetivos ya estuvieron claros, se pasó a la elección de tecnologías que se iban a usar. La decisión fue tomada en base la experiencia, ya que de esta manera se conseguía rápidamente un prototipo manejable. 

Para el Frontend, se decidió el uso de HTML, CSS y JavaScript, debido la posibilidad de creación de interfaces de usuario fluidas y dinámicas. En cambio, para el Backend se usó Flask, por su fácil adaptabilidad con diversas librerías de Python. Para la gestión de la base de datos, se optó por PostgreSQL, ya que garantizaba una estructura robusta y escalable.

\subsection {Fase de desarrollo}

La fase de desarrollo se centró en implementar las características clave identificadas durante la fase anterior. Las principales características desarrolladas fueron:

\begin{enumerate}
    \item \textbf{Corrección automatizada de los ejercicios}: Se desarrolló un sistema de corrección automatizada que no solo verifica la solución del ejercicio, sino también la calidad y eficiencia del código. Concretamente, analiza la limpieza del código, la utilización correcta de las estructuras de datos y si existen bucles repetitivos o redundantes. Garantizando que los estudiantes no solo resuelven el problema, sino que también adopten buenas prácticas de codificación.

    \item \textbf{Retroalimentación Personaliza}: Se debe ofrecer un buen \textit{feedback} para que los estudiantes entiendan sus errores. Por ello, se proporcionan comentarios y algunas sugerencias necesarias. De esta manera, se fortalece la comprensión y se facilita el aprendizaje.

    \item \textbf{Caminos de Ejercicios Aleatorio}: Para poder proporcionar una buena experiencia de aprendizaje, el sistema tiene un conjunto de ejercicios basados en requisitos específicos de forma aleatoria. De esta manera, se consigue que cada vez que el estudiante se enfrenta un requisito, se le desafía con un conjunto de ejercicios diferentes, asegurando que los haga por comprensión y no por repetición. Es importante destacar, que si un estudiante no realiza correctamente un ejercicio, el sistema le proporcionará otro diferente para evitar frustración.
\end{enumerate}

Cabe destacar que esta fase también implicó una serie de pruebas iterativas y correcciones, permitiendo ajustes y mejoras en tiempo real en función del feedback recibido y las observaciones realizadas.

\subsection{Fase de prueba y corrección}

Tras finalizar la anterior fase, tocó centrarse en el testeo del prototipo y sus respectivos ajustes. A pesar de haber diseñado la página web en base la investigación teórica, se debía validar su usabilidad y efectividad de las funciones realizadas.

Se eligió un grupo de 2-3 estudiantes, para que interactuaran con la plataforma. 
Pidiéndoles que hicieran varias tareas y que compartieran sus impresiones, dificultades y recomendaciones.

La retroalimentación de estos estudiantes fue clave. A partir de sus comentarios, se pudieron efectuar mejoras importantes en la plataforma. Se solucionaron fallos técnicos y cuestiones de usabilidad, además de optimizar la calidad de la experiencia educativa. De esta manera, se consiguió que la plataforma se adaptará a las necesidades y retos de los estudiantes mejor, pudiendo asegurar una experiencia educativa de calidad.

\subsection{Elaboración de la memoria}

Durante las fases previas del proyecto, se llevó a cabo una recolección de datos e información pertinente. Al aproximarnos a las etapas finales del desarrollo, dicha información fue estructurada y articulada de manera coherente y sistemática, siendo la creación de la memoria.

El proyecto tuvo inicio a principios de julio, durando cinco meses hasta su finalización. Es importante subrayar que el período previo a julio, aunque incluyó actividades de investigación y lectura, no es considerado crucial para la estructura del proyecto. Esto se debe a que, a pesar de haber realizado dichas tareas, no se le dedicó una cantidad de tiempo significante debido a otras responsabilidades. Por esta razón, se presenta el diagrama de Gantt \ref{fig:gantt_diagram} que refleja el periodo de desarrollo del proyecto:

\begin{figure}[H]
    \centering
    \scriptsize
    \resizebox{\textwidth}{!}{%
    \begin{ganttchart}[
        hgrid,
        vgrid={*{6}{draw=none}, dotted},
        title/.append style={fill=gray!30},
        title label font=\tiny,
        title height=1,
        bar/.append style={fill=blue!30},
        bar height=.4,
        y unit chart=0.5cm,
        x unit=0.12cm,
        time slot format=isodate,
        milestone left shift =-1,
        milestone right shift =2
    ]{2023-07-01}{2023-11-14}
    
    \gantttitlecalendar{month=name} \\
    
    % Tareas
    \ganttbar{Reuniones con el Tutor}{2023-07-31}{2023-07-31}
    \ganttbar{}{2023-09-11}{2023-09-11}
    \ganttbar{}{2023-09-18}{2023-09-18}
    \ganttbar{}{2023-09-25}{2023-09-25} 
    \ganttbar{}{2023-10-02}{2023-10-02}
    \ganttbar{}{2023-10-09}{2023-10-09}
    \ganttbar{}{2023-10-16}{2023-10-16}
    \ganttbar{}{2023-10-23}{2023-10-23}
    \ganttbar{}{2023-10-31}{2023-10-31}
    \ganttbar{}{2023-11-06}{2023-11-06} \\
    
    \ganttgroup{Investigación}{2023-07-01}{2023-07-17} \\
    \ganttbar{ITS}{2023-07-01}{2023-07-10} \\
    \ganttbar{Pedagogía de la programación}{2023-07-11}{2023-07-15} \\
    \ganttbar{Herramientas actuales}{2023-07-15}{2023-07-17} \\
    
    \ganttgroup{Definición}{2023-07-16}{2023-07-31} \\
    \ganttbar{Definición y análisis del proyecto}{2023-07-16}{2023-07-24} \\
    \ganttbar{Estructuración del contenido}{2023-07-25}{2023-07-31} \\
    
    \ganttgroup{Definición y creación BBDD}{2023-08-01}{2023-08-10} \\
    
    \ganttgroup{Desarrollo Frontend}{2023-08-10}{2023-08-24} \\
    \ganttbar{Páginas del estudiante}{2023-08-10}{2023-08-15} \\
    \ganttbar{Página del tutor}{2023-08-15}{2023-08-19} \\
    \ganttbar{Página del administrador}{2023-08-19}{2023-08-24} \\
    
    \ganttgroup{Desarrollo Backend}{2023-09-01}{2023-09-25} \\
    \ganttbar{Funciones}{2023-09-01}{2023-09-08} \\
    \ganttbar{Editor de código}{2023-09-08}{2023-09-16} \\
    \ganttbar{Batch de ejercicios}{2023-09-17}{2023-09-21} \\
    \ganttbar{Pruebas}{2023-09-22}{2023-09-25} \\
    
    \ganttgroup{Módulo ITS}{2023-09-25}{2023-10-10} \\
    \ganttbar{Definición estructura contenidos}{2023-09-25}{2023-09-30} \\
    \ganttbar{Corrección automatizada}{2023-10-01}{2023-10-06} \\
    \ganttbar{Valoración conocimientos}{2023-10-07}{2023-10-10} \\
    
    \ganttgroup{Integración ITS-WEB}{2023-10-10}{2023-10-14} \\
    \ganttgroup{Pruebas y Evaluación}{2023-10-14}{2023-10-16} \\
    \ganttgroup{Mejoras prototipo}{2023-10-17}{2023-10-22}
    \ganttgroup{}{2023-11-01}{2023-11-06} \\
    
    \ganttgroup{Documentación}{2023-10-22}{2023-11-13} \\
    \ganttbar{Redacción documentación}{2023-10-22}{2023-10-31} \\
    \ganttbar{Revisión y correcciones}{2023-11-04}{2023-11-13} \\
    \ganttgroup{Despliegue}{2023-11-10}{2023-11-11}
    \end{ganttchart}
    }
    \caption{Diagrama de Gantt con la planificación del TFG}
    \label{fig:gantt_diagram}
\end{figure}

\section{Presupuesto}

En esta parte del documento se realiza una descripción de los gastos ocasionados por el proyecto, formando el denominado presupuesto \ref{tab:resumen-costes}.

\subsection{Licencias}

Con tal de realizar una buena evaluación del coste de un proyecto, se tuvieron en cuenta todos aquellos gastos relacionados con licencias. Por ello, se optó por maximizar el uso de tecnologías \textit{OpenSource} para reducir costos.

Para la interfaz de usuario, se optó por usar HTML, CSS y JavaScript por su costo cero. En cuanto al Backend, se eligió Flask por el mismo motivo. Además, se recurrió a PostgreSQL que no tiene ningún coste ni límite en cuanto a tamaño de la base de datos. Para el control de versiones, se usó Git y GitHub \cite{personalgithub}. Aunque GitHub ofrece planes de pago, dispone de buenas opciones gratuitas. 

Por lo tanto, el coste total de las licencias es cero. 

\subsection{Recursos Materiales}

Dentro de la inversión material, es indispensable considerar el equipo informático empleado en el proyecto, en este caso, un ordenador portátil tal como se puede ver en el apéndice \ref{chap:recursos}. Comúnmente, estos dispositivos tienen una vida útil estimada que oscila entre los 2 y 4 años. 

Si asumimos una vida útil intermedia de 3 años para el dispositivo, se estimará una depreciación anual del equipo. Por ello, considerando que el portátil tuvo un precio inicial de 950€, su depreciación anual será de 316,67€. 

Finalmente, se debe tener en cuenta que el proyecto se ha desarrollado en una duración aproximada de 5 meses, por ello, el coste asociado al desgaste del portátil durante dicho período es de 132,78€.

\subsection{Costes de Personal}

Para el cálculo de costes de personal se ha tenido en cuenta un único empleado: un ingeniero informático. Tomando los datos obtenidos de la Seguridad Social para el año 2023 \cite{seg-social}, el salario de dicho ingeniero oscila entre una base mínima de cotización de 1.759,50€ y una base máxima de 4.495,50€ mensuales \ref{tab:coste-personal}.

\begin{table}[H]
    \centering
    \small 
    \setlength{\tabcolsep}{1.8pt}
    \begin{tabular}{@{}lrr@{}}
        \toprule
        \textbf{Concepto} & \textbf{Coste Mensual (€)} & \textbf{Coste 5 Meses (€)} \\
        \midrule
        Base Mínima de Cotización & 1.759,50 & 8.797,50 \\
        Cotización Empresa (Base Mínima) & 415,44 & 2.077,20 \\
        Cotización Trabajador (Base Mínima) & 82,70 & 413,50 \\
        \textbf{Total (Base Mínima)} & \textbf{2.257,64} & \textbf{11.288,20} \\
        \midrule
        Base Máxima de Cotización & 4.495,50 & 22.477,50 \\
        Cotización Empresa (Base Máxima) & 1.060,94 & 5.304,70 \\
        Cotización Trabajador (Base Máxima) & 211,29 & 1.056,45 \\
        \textbf{Total (Base Máxima)} & \textbf{5.767,73} & \textbf{28.838,65} \\
        \bottomrule
    \end{tabular}

    \caption{Costes asociados al personal para un ingeniero informático.}
    \label{tab:coste-personal}

\end{table}

Si calculamos el promedio salarial, se obtendría aproximadamente unos 3.127,50€. Como el proyecto tiene una duración de 5 meses, el coste total en salarios sería de 15.637,50€ para el desarrollo completo del proyecto.

Adicionalmente, se debería considerar las cotizaciones correspondientes a la empresa y al trabajador, siendo el coste total para el proyecto de unos 17.900€ aproximadamente. 

\subsection{Otros costes}

Dentro de este apartado se hace alusión a todos aquellos gastos indirectos importantes para la realización del proyecto. Por ejemplo, el material de oficina, la conexión a Internet, la electricidad, los gastos de transporte, entre otros.

Este tipo de costes indirectos se suele estimar como un porcentaje del gasto principal, en este caso, del coste de personal. Siguiendo esta norma, se ha determinado que estos gastos representan aproximadamente el 10\% del gasto total en personal:

\begin{equation}
17.900€ \times 0,10 = 1.790€
\end{equation}

Por lo tanto, se estima un coste adicional de 1.790€ para cubrir lo gastos indirectos asociados al proyecto.

\subsection{Total}

Para finalizar, se muestra una tabla \ref{tab:resumen-costes} que resume todos los costes obtenidos, haciendo una estimación mensual y total del proyecto.
\begin{table}[H]
    \centering
    \resizebox{\textwidth}{!}{%
    \begin{tabular}{lcc}
        \hline
        \textbf{Concepto} & \textbf{Coste Mensual (€)} & \textbf{Coste Total (5 Meses) (€)} \\
        \hline
        Licencias & 0,00 & 0,00 \\
        Recursos Materiales (Portátil) & 26,56 & 132,78 \\
        Personal (Salario promedio) & 3.127,50 & 15.637,50 \\
        Cotizaciones (aprox.) & 452,50 & 2.262,50 \\
        Otros costes & 358,00 & 1.790,00 \\
        \hline
        \textbf{Total} & \textbf{3964,06} & \textbf{19.822,78} \\
        \hline
    \end{tabular}
    }
    \caption{Total de costes del proyecto}
    \label{tab:resumen-costes}
\end{table}
