\chapter{Requisitos} \label{chap:requisitos}

Basado en el análisis presentado anteriormente, este capítulo se dedica a perfilar los requisitos de la plataforma. La evaluación previa ha establecido un fundamento detallar dichos requisitos. Estos serán el eje que finalizará el proceso de análisis y dirigirá el proceso de desarrollo, con el propósito de garantizar que la plataforma resultante satisfaga plenamente las expectativas y requisitos de los usuarios finales.

\section{Requisitos Funcionales}

Con tal de garantizar la funcionalidad del programa y la realización de los objetivos establecidos, es esencial tener una buena definición de los requisitos funcionales. Para facilitar su comprensión, presentaremos estos requisitos en tablas, siguiendo las convenciones de la ingeniería de requisitos y desarrollo de software. La organización de los contenidos es la siguiente:

\begin{itemize}
    \item \textbf{ID:} Identificador único para cada requisito. Esencial para una fácil referencia y seguimiento.
    \item \textbf{Nombre:} Nombre del requisito.
    \item \textbf{Descripción:} Explicación del propósito y alcance del requisito.
    \item \textbf{Prioridad:} La importancia del requisito en relación con el sistema. Las categorías posibles son \textit{alta}, \textit{media} o \textit{baja}. Siendo crucial abordar primero los requisitos de alta prioridad, mientras que los de baja prioridad se harán al final.
    \item \textbf{Estado:} Refleja la situación actual del requisito, pudiendo estar \textit{en desarrollo}, \textit{completado}, \textit{en revisión} o \textit{rechazado}.
\end{itemize}


\subsection{Generales}

\begin{table}[H]
    \centering
    \begin{tabular}{|l|p{9.5cm}|}
        \hline
        \textbf{ID} & RF-01 \\
        \hline
        \textbf{Nombre} & Inicio de sesión \\
        \hline
        \textbf{Descripción} & Los usuarios registrados deben poder acceder a la plataforma introduciendo su nombre de usuario y contraseña. El sistema debe garantizar la confidencialidad y autenticidad de esta operación, asegurando que las contraseñas se gestionan y almacenan de manera segura. \\
        \hline
        \textbf{Prioridad} & Alta \\
        \hline
        \textbf{Estado} & Completado \\
        \hline
    \end{tabular}
    \caption{Requisito funcional RF-01: Inicio de sesión.}
    \label{table:req-RF001}
\end{table}

\begin{table}[H]
    \centering
    \begin{tabular}{|l|p{9.5cm}|}
        \hline
        \textbf{ID} & RF-02 \\
        \hline
        \textbf{Nombre} & Registro de usuario \\
        \hline
        \textbf{Descripción} & Los nuevos usuarios deben poder registrarse en la plataforma proporcionando información relevante. \\
        \hline
        \textbf{Prioridad} & Alta \\
        \hline
        \textbf{Estado} & Completado \\
        \hline
    \end{tabular}
    \caption{Requisito funcional RF-02: Registro de usuario.}
    \label{table:req-RF002}
\end{table}

\begin{table}[H]
    \centering
    \begin{tabular}{|l|p{9.5cm}|}
        \hline
        \textbf{ID} & RF-03 \\
        \hline
        \textbf{Nombre} & Perfil de usuario \\
        \hline
        \textbf{Descripción} & Cada usuario debe tener un perfil donde pueda visualizar su información personal, y donde poder cambiar ciertos datos.\\
        \hline
        \textbf{Prioridad} & Media \\
        \hline
        \textbf{Estado} & Completado \\
        \hline
    \end{tabular}
    \caption{Requisito funcional RF-03: Perfil de usuario.}
    \label{table:req-RF005}
\end{table}
    

\subsection{Sistema}
        
\begin{table}[H]
    \centering
    \begin{tabular}{|l|p{9.5cm}|}
        \hline
        \textbf{ID} & RF-04 \\
        \hline
        \textbf{Nombre} & Selección de ejercicios por requisitos \\
        \hline
        \textbf{Descripción} & El sistema deberá presentar a los estudiantes ejercicios basados en el requisito que estén estudiando. Una vez completado un ejercicio, el siguiente deberá ser mostrado de forma aleatoria dentro del mismo requisito, hasta que se considere que el estudiante ha asimilado ese requisito. \\
        \hline
        \textbf{Prioridad} & Alta \\
        \hline   
        \textbf{Estado} & Completado \\
        \hline
    \end{tabular}
    \caption{Requisito funcional RF-04: Selección de ejercicios por requisitos.}
    \label{table:req-RF003}
\end{table}

\begin{table}[H]
    \centering
    \begin{tabular}{|l|p{9.5cm}|}
        \hline
        \textbf{ID} & RF-05 \\
        \hline
        \textbf{Nombre} & Corrección de ejercicios \\
        \hline
        \textbf{Descripción} & Al enviar una solución, el sistema no solo verificará si es correcta, sino que también evaluará la calidad del código en términos de optimización y buenas prácticas de programación. \\
        \hline
        \textbf{Prioridad} & Alta \\
        \hline
        \textbf{Estado} & Completado \\
        \hline
    \end{tabular}
    \caption{Requisito funcional RF-05: Corrección de ejercicios.}
    \label{table:req-RF004}
\end{table}
        
\begin{table}[H]
    \centering
    \begin{tabular}{|l|p{9.5cm}|}
        \hline
        \textbf{ID} & RF-06 \\
        \hline
        \textbf{Nombre} & Compilador integrado \\
        \hline
        \textbf{Descripción} & La plataforma deberá contener con un compilador integrado que permita a los estudiantes probar sus códigos en Java, HTML+CSS+JS, Python y C++. \\
        \hline
        \textbf{Prioridad} & Alta \\
        \hline
        \textbf{Estado} & Completado \\
        \hline
    \end{tabular}
    \caption{Requisito funcional RF-06: Compilador integrado.}
    \label{table:req-RF007}
\end{table}

\begin{table}[H]
    \centering
    \begin{tabular}{|l|p{9.5cm}|}
        \hline
        \textbf{ID} & RF-07 \\
        \hline
        \textbf{Nombre} & Categorización de la teoría \\
        \hline
        \textbf{Descripción} & La teoría se clasificará en función del requisito, facilitando así su visualización dependiendo del punto en que se encuentre el estudiante. \\
        \hline
        \textbf{Prioridad} & Media \\
        \hline
        \textbf{Estado} & Completado \\
        \hline
    \end{tabular}
    \caption{Requisito funcional RF-07: Categorización de la teoría.}
    \label{table:req-RF008}
\end{table}

\begin{table}[H]
    \centering
    \begin{tabular}{|l|p{9.5cm}|}
        \hline
        \textbf{ID} & RF-08 \\
        \hline
        \textbf{Nombre} & Repetición de ejercicios similares tras fallo \\
        \hline
        \textbf{Descripción} & Si un estudiante falla un ejercicio, el sistema deberá presentar otro ejercicio con características similares para que el estudiante pueda intentarlo de nuevo. \\
        \hline
        \textbf{Prioridad} & Alta \\
        \hline
        \textbf{Estado} & Completado \\
        \hline
    \end{tabular}
    \caption{Requisito funcional RF-08: Repetición de ejercicios similares tras fallo.}
    \label{table:req-RF00X}
\end{table}

\begin{table}[H]
    \centering
    \begin{tabular}{|l|p{9.5cm}|}
        \hline
        \textbf{ID} & RF-09 \\
        \hline
        \textbf{Nombre} & Límite de tiempo para completar un ejercicio \\
        \hline
        \textbf{Descripción} & Si el estudiante tarda más de un tiempo predefinido en completar un ejercicio, el sistema lo marcará como erróneo y presentará un nuevo ejercicio. \\
        \hline
        \textbf{Prioridad} & Media \\
        \hline
        \textbf{Estado} & Completado \\
        \hline
    \end{tabular}
    \caption{Requisito funcional RF-09: Límite de tiempo para completar un ejercicio.}
    \label{table:req-RF00Y}
\end{table}

\subsection{Estudiantes}

\begin{table}[H]
    \centering
    \begin{tabular}{|l|p{9.5cm}|}
        \hline
        \textbf{ID} & RF-10 \\
        \hline
        \textbf{Nombre} & Resolución de dudas \\
        \hline
        \textbf{Descripción} & Los estudiantes realizarán preguntas y podrán adjuntar archivos para que el profesor les ayude. \\
        \hline
        \textbf{Prioridad} & Alta \\
        \hline
        \textbf{Estado} & Completado \\
        \hline
    \end{tabular}
    \caption{Requisito funcional RF-10: Resolución de dudas.}
    \label{table:req-RF006}
\end{table}

\begin{table}[H]
    \centering
    \begin{tabular}{|l|p{9.5cm}|}
        \hline
        \textbf{ID} & RF-11 \\
        \hline
        \textbf{Nombre} & Consulta de los \textit{rankings} \\
        \hline
        \textbf{Descripción} & Los estudiantes podrán consultar un ranking donde vean los \textit{top} 5 estudiantes diarios, semanales y mensuales. \\
        \hline
        \textbf{Prioridad} & Media \\
        \hline
        \textbf{Estado} & Completado \\
        \hline
    \end{tabular}
    \caption{Requisito funcional RF-11: Consulta de los rankings.}
    \label{table:req-RF00Z}
\end{table}

\begin{table}[H]
    \centering
    \begin{tabular}{|l|p{9.5cm}|}
        \hline
        \textbf{ID} & RF-12 \\
        \hline
        \textbf{Nombre} & Juegos en la web \\
        \hline
        \textbf{Descripción} & Por cada módulo que los estudiantes completen, se les desbloqueará un juego como recompensa. \\
        \hline
        \textbf{Prioridad} & Baja \\
        \hline
        \textbf{Estado} & Completado \\
        \hline
    \end{tabular}
    \caption{Requisito funcional RF-12: Juegos en la web.}
    \label{table:req-RF00A}
\end{table}

\begin{table}[H]
    \centering
    \begin{tabular}{|l|p{9.5cm}|}
        \hline
        \textbf{ID} & RF-13 \\
        \hline
        \textbf{Nombre} & Consulta de los ejercicios corregidos \\
        \hline
        \textbf{Descripción} & Los estudiantes podrán consultar los comentarios del profesor sobre los ejercicios que haya corregido. \\
        \hline
        \textbf{Prioridad} & Media \\
        \hline
        \textbf{Estado} & Completado \\
        \hline
    \end{tabular}
    \caption{Requisito funcional RF-13: Consulta de los ejercicios corregidos.}
    \label{table:req-RF00B}
\end{table}

\begin{table}[H]
    \centering
    \begin{tabular}{|l|p{9.5cm}|}
        \hline
        \textbf{ID} & RF-14 \\
        \hline
        \textbf{Nombre} & Sistema de notificación \\
        \hline
        \textbf{Descripción} & Se notificará al estudiante si su ejercicio es correcto o incorrecto cuando lo evalúe el profesor, junto con una retroalimentación del por qué es incorrecto. \\
        \hline
        \textbf{Prioridad} & Alta \\
        \hline
        \textbf{Estado} & Completado \\
        \hline
    \end{tabular}
    \caption{Requisito funcional RF-14: Sistema de notificación.}
    \label{table:req-RF00C}
\end{table}

\begin{table}[H]
    \centering
    \begin{tabular}{|l|p{9.5cm}|}
        \hline
        \textbf{ID} & RF-15 \\
        \hline
        \textbf{Nombre} & Feedback sobre la solución \\
        \hline
        \textbf{Descripción} & Cuando un alumno envíe la solución, se le mostrará un feedback completo sobre ella. \\
        \hline
        \textbf{Prioridad} & Alta \\
        \hline
        \textbf{Estado} & Completado \\
        \hline
    \end{tabular}
    \caption{Requisito funcional RF-15: Feedback sobre la solución.}
    \label{table:req-RF00D}
\end{table}

\begin{table}[H]
    \centering
    \begin{tabular}{|l|p{9.5cm}|}
        \hline
        \textbf{ID} & RF-16 \\
        \hline
        \textbf{Nombre} & Sistema de puntos \\
        \hline
        \textbf{Descripción} & Por cada ejercicio completado, el estudiante obtendrá una cantidad de puntos dependiendo de la calidad de su solución. \\
        \hline
        \textbf{Prioridad} & Baja \\
        \hline
        \textbf{Estado} & Completado \\
        \hline
    \end{tabular}
    \caption{Requisito funcional RF-16: Sistema de puntos.}
    \label{table:req-RF00E}
\end{table}

\subsection{Profesores}

\begin{table}[H]
    \centering
    \begin{tabular}{|l|p{9.5cm}|}
        \hline
        \textbf{ID} & RF-17 \\
        \hline
        \textbf{Nombre} & Lista de usuarios \\
        \hline
        \textbf{Descripción} & El profesor podrá ver una lista de todos los usuarios registrados en la plataforma. \\
        \hline
        \textbf{Prioridad} & Media \\
        \hline
        \textbf{Estado} & Completado \\
        \hline
    \end{tabular}
    \caption{Requisito funcional RF-17: Lista de usuarios.}
    \label{table:req-RF00F}
\end{table}

\begin{table}[H]
    \centering
    \begin{tabular}{|l|p{9.5cm}|}
        \hline
        \textbf{ID} & RF-18 \\
        \hline
        \textbf{Nombre} & Consulta de ejercicios realizados \\
        \hline
        \textbf{Descripción} & El profesor podrá consultar los ejercicios realizados por los estudiantes, junto con su estado, \\
        \hline
        \textbf{Prioridad} & Media \\
        \hline
        \textbf{Estado} & Completado \\
        \hline
    \end{tabular}
    \caption{Requisito funcional RF-18: Consulta de ejercicios realizados.}
    \label{table:req-RF00G}
\end{table}


\begin{table}[H]
    \centering
    \begin{tabular}{|l|p{9.5cm}|}
        \hline
        \textbf{ID} & RF-19 \\
        \hline
        \textbf{Nombre} & Panel de resolución de preguntas \\
        \hline
        \textbf{Descripción} & El profesor tendrá un panel donde podrá responder las preguntas realizadas por los estudiantes. \\
        \hline
        \textbf{Prioridad} & Alta \\
        \hline
        \textbf{Estado} & Completado \\
        \hline
    \end{tabular}
    \caption{Requisito funcional RF-19: Panel de resolución de preguntas.}
    \label{table:req-RF00H}
\end{table}

\begin{table}[H]
    \centering
    \begin{tabular}{|l|p{9.5cm}|}
        \hline
        \textbf{ID} & RF-20 \\
        \hline
        \textbf{Nombre} & Lista de estudiantes inactivos \\
        \hline
        \textbf{Descripción} & El profesor podrá consultar en una lista los estudiantes que llevan más de una semana sin conectarse, incluyendo sus correos electrónicos para un posible seguimiento. \\
        \hline
        \textbf{Prioridad} & Media \\
        \hline
        \textbf{Estado} & Completado \\
        \hline
    \end{tabular}
    \caption{Requisito funcional RF-20: Lista de estudiantes inactivos.}
    \label{table:req-RF00I}
\end{table}

\begin{table}[H]
    \centering
    \begin{tabular}{|l|p{9.5cm}|}
        \hline
        \textbf{ID} & RF-21\\
        \hline
        \textbf{Nombre} & Tiempo de resolución de ejercicios \\
        \hline
        \textbf{Descripción} & Habrá una lista de estudiantes que tengan un tiempo de resolución de ejercicios demasiado bajo o alto, con tal de identificar posibles problemas o áreas de mejora. \\
        \hline
        \textbf{Prioridad} & Baja \\
        \hline
        \textbf{Estado} & Completado \\
        \hline
    \end{tabular}
    \caption{Requisito funcional RF-21: Tiempo de resolución de ejercicios.}
    \label{table:req-RF00J}
\end{table}

\begin{table}[H]
    \centering
    \begin{tabular}{|l|p{9.5cm}|}
        \hline
        \textbf{ID} & RF-22 \\
        \hline
        \textbf{Nombre} & Corrección de ejercicios \\
        \hline
        \textbf{Descripción} & El profesor tendrá la capacidad de corregir los ejercicios, relacionados con HTML, enviados por los estudiantes y ser capaz de proporcionar una retroalimentación. \\
        \hline
        \textbf{Prioridad} & Alta \\
        \hline
        \textbf{Estado} & Completado \\
        \hline
    \end{tabular}
    \caption{Requisito funcional RF-22: Corrección de ejercicios.}
    \label{table:req-RF00K}
\end{table}

\begin{table}[H]
    \centering
    \begin{tabular}{|l|p{9.5cm}|}
        \hline
        \textbf{ID} & RF-23 \\
        \hline
        \textbf{Nombre} & Usuarios con múltiples fallos \\
        \hline
        \textbf{Descripción} & Se listarán los usuarios que han fallado más de dos veces un ejercicio específico con tal de poder identificar áreas problemáticas. \\
        \hline
        \textbf{Prioridad} & Baja \\
        \hline
        \textbf{Estado} & Completado \\
        \hline
    \end{tabular}
    \caption{Requisito funcional RF-23: Usuarios con múltiples fallos.}
    \label{table:req-RF00L}
\end{table}

\begin{table}[H]
    \centering
    \begin{tabular}{|l|p{9.5cm}|}
        \hline
        \textbf{ID} & RF-24 \\
        \hline
        \textbf{Nombre} & Ejercicios realizados por día \\
        \hline
        \textbf{Descripción} & Habrá un gráfico donde se podrá consultar el número de ejercicios realizados por día en la plataforma. Con tal de poder evaluar la actividad y el compromiso de los estudiantes. \\
        \hline
        \textbf{Prioridad} & Baja \\
        \hline
        \textbf{Estado} & Completado \\
        \hline
    \end{tabular}
    \caption{Requisito funcional RF-24: Ejercicios realizados por día.}
    \label{table:req-RF00N}
\end{table}

\begin{table}[H]
    \centering
    \begin{tabular}{|l|p{9.5cm}|}
        \hline
        \textbf{ID} & RF-25 \\
        \hline
        \textbf{Nombre} & Usuarios con alta tasa de errores \\
        \hline
        \textbf{Descripción} & Habrá una lista de usuarios con una alta tasa de errores en los ejercicios, con tal de poder ofrecer asistencia adicional. \\
        \hline
        \textbf{Prioridad} & Baja \\
        \hline
        \textbf{Estado} & Completado \\
        \hline
    \end{tabular}
    \caption{Requisito funcional RF-25: Usuarios con alta tasa de errores.}
    \label{table:req-RF00M}
\end{table}

\subsection{Administrador}

\begin{table}[H]
    \centering
    \begin{tabular}{|l|p{9.5cm}|}
        \hline
        \textbf{ID} & RF-26 \\
        \hline
        \textbf{Nombre} & Alta de profesor por administrador \\
        \hline
        \textbf{Descripción} & El administrador tendrá la capacidad de registrar a un profesor mediante el panel de administración. \\
        \hline
        \textbf{Prioridad} & Alta \\
        \hline
        \textbf{Estado} & Completado \\
        \hline
    \end{tabular}
    \caption{Requisito funcional RF-26: Alta de profesor por administrador.}
    \label{table:req-RF0010}
\end{table}
        
\begin{table}[H]
    \centering
    \begin{tabular}{|l|p{9.5cm}|}
        \hline
        \textbf{ID} & RF-27 \\
        \hline
        \textbf{Nombre} & Añadir elementos \\
        \hline
        \textbf{Descripción} & El administrador podrá añadir nuevos módulos, requisitos, profesores, teoría y ejercicios a la plataforma. \\
        \hline
        \textbf{Prioridad} & Alta \\
        \hline
        \textbf{Estado} & Completado \\
        \hline
    \end{tabular}
    \caption{Requisito funcional RF-27: Añadir elementos.}
    \label{table:req-RF00O}
\end{table}

\begin{table}[H]
    \centering
    \begin{tabular}{|l|p{9.5cm}|}
        \hline
        \textbf{ID} & RF-28 \\
        \hline
        \textbf{Nombre} & Consultar elementos \\
        \hline
        \textbf{Descripción} & El administrador podrá consultar los detalles de los módulos, requisitos, profesores, teoría y ejercicios existentes en la plataforma. \\
        \hline
        \textbf{Prioridad} & Alta \\
        \hline
        \textbf{Estado} & Completado \\
        \hline
    \end{tabular}
    \caption{Requisito funcional RF-28: Consultar elementos.}
    \label{table:req-RF00P}
\end{table}

\begin{table}[H]
    \centering
    \begin{tabular}{|l|p{9.5cm}|}
        \hline
        \textbf{ID} & RF-29 \\
        \hline
        \textbf{Nombre} & Editar elementos \\
        \hline
        \textbf{Descripción} & El administrador podrá editar los detalles de los módulos, profesores, teoría y ejercicios, excepto los requisitos que no son editables. \\
        \hline
        \textbf{Prioridad} & Alta \\
        \hline
        \textbf{Estado} & Completado \\
        \hline
    \end{tabular}
    \caption{Requisito funcional RF-29: Editar elementos.}
    \label{table:req-RF00Q}
\end{table}

\begin{table}[H]
    \centering
    \begin{tabular}{|l|p{9.5cm}|}
        \hline
        \textbf{ID} & RF-30 \\
        \hline
        \textbf{Nombre} & Eliminar elementos \\
        \hline
        \textbf{Descripción} & El administrador podrá eliminar módulos, requisitos, profesores, teoría y ejercicios de la plataforma. \\
        \hline
        \textbf{Prioridad} & Alta \\
        \hline
        \textbf{Estado} & Completado \\
        \hline
    \end{tabular}
    \caption{Requisito funcional RF-30: Eliminar elementos.}
    \label{table:req-RF00R}
\end{table}

\begin{table}[H]
    \centering
    \begin{tabular}{|l|p{9.5cm}|}
        \hline
        \textbf{ID} & RF-31 \\
        \hline
        \textbf{Nombre} & Orden global \\
        \hline
        \textbf{Descripción} & El administrador podrá crear y consultar un orden global que especifica el camino que debe seguir el estudiante en la plataforma, siendo una secuencia de módulos y requisitos. \\
        \hline
        \textbf{Prioridad} & Alta \\
        \hline
        \textbf{Estado} & Completado \\
        \hline
    \end{tabular}
    \caption{Requisito funcional RF-31: Orden global.}
    \label{table:req-RF00S}
\end{table}


\section{Requisitos No Funcionales}

Con tal de asegurar la calidad global del sistema, deberemos tener en cuenta los requisitos no funcionales. Por ello, se van a detallar estos requisitos de manera estructurada siguiendo el mismo formato que con los requisitos funcionales.

\begin{table}[H]
    \centering
    \begin{tabular}{|l|p{9.5cm}|}
        \hline
        \textbf{ID} & RNF-01 \\
        \hline
        \textbf{Nombre} & Usabilidad \\
        \hline
        \textbf{Descripción} & El sistema deberá ser intuitivo y fácil de usar para los diferentes tipos de usuarios. \\
        \hline
        \textbf{Prioridad} & Alta \\
        \hline
        \textbf{Estado} & Completado \\
        \hline
    \end{tabular}
    \caption{Requisito no funcional RNF-01: Usabilidad.}
    \label{table:req-RNF31}
\end{table}

\begin{table}[H]
    \centering
    \begin{tabular}{|l|p{9.5cm}|}
        \hline
        \textbf{ID} & RNF-02 \\
        \hline
        \textbf{Nombre} & Rendimiento \\
        \hline
        \textbf{Descripción} & El sistema deberá ser capaz de manejar una cantidad promedio de usuarios. \\
        \hline
        \textbf{Prioridad} & Media \\
        \hline
        \textbf{Estado} & Completado \\
        \hline
    \end{tabular}
    \caption{Requisito no funcional RNF-02: Rendimiento.}
    \label{table:req-RNF32}
\end{table}

\begin{table}[H]
    \centering
    \begin{tabular}{|l|p{9.5cm}|}
        \hline
        \textbf{ID} & RNF-03 \\
        \hline
        \textbf{Nombre} & Seguridad \\
        \hline
        \textbf{Descripción} & Todos los datos sensibles, como contraseñas, deben estar encriptados. \\
        \hline
        \textbf{Prioridad} & Alta \\
        \hline
        \textbf{Estado} & Completado \\
        \hline
    \end{tabular}
    \caption{Requisito no funcional RNF-03: Seguridad.}
    \label{table:req-RNF33}
\end{table}

\begin{table}[H]
    \centering
    \begin{tabular}{|l|p{9.5cm}|}
        \hline
        \textbf{ID} & RNF-04 \\
        \hline
        \textbf{Nombre} & Compatibilidad \\
        \hline
        \textbf{Descripción} & El sistema debe ser compatible con los navegadores web más comunes, incluidos Chrome, Firefox y Safari. \\
        \hline
        \textbf{Prioridad} & Alta \\
        \hline
        \textbf{Estado} & Completado \\
        \hline
    \end{tabular}
    \caption{Requisito no funcional RNF-04: Compatibilidad.}
    \label{table:req-RNF35}
\end{table}

\begin{table}[H]
    \centering
    \begin{tabular}{|l|p{9.5cm}|}
        \hline
        \textbf{ID} & RNF-05 \\
        \hline
        \textbf{Nombre} & Escalabilidad \\
        \hline
        \textbf{Descripción} & El sistema deberá ser escalable para permitir la adición de más módulos, ejercicios y usuarios en el futuro. \\
        \hline
        \textbf{Prioridad} & Alta \\
        \hline
        \textbf{Estado} & Completado \\
        \hline
    \end{tabular}
    \caption{Requisito no funcional RNF-05: Escalabilidad.}
    \label{table:req-RNF36}
\end{table}

\begin{table}[H]
    \centering
    \begin{tabular}{|l|p{9.5cm}|}
        \hline
        \textbf{ID} & RNF-06 \\
        \hline
        \textbf{Nombre} & Disponibilidad \\
        \hline
        \textbf{Descripción} & La página web debe estar disponible el 99\% del tiempo, excluyendo el tiempo de mantenimiento programado. \\
        \hline
        \textbf{Prioridad} & Media \\
        \hline
        \textbf{Estado} & Completado \\
        \hline
    \end{tabular}
    \caption{Requisito no funcional RNF-06: Disponibilidad.}
    \label{table:req-RNF37}
\end{table}

\begin{table}[H]
    \centering
    \begin{tabular}{|l|p{9.5cm}|}
        \hline
        \textbf{ID} & RNF-07 \\
        \hline
        \textbf{Nombre} & Tiempo de Respuesta \\
        \hline
        \textbf{Descripción} & El tiempo de carga de las páginas no debe superar los 5 segundos. \\
        \hline
        \textbf{Prioridad} & Media \\
        \hline
        \textbf{Estado} & Completado \\
        \hline
    \end{tabular}
    \caption{Requisito no funcional RNF-07: Tiempo de Respuesta.}
    \label{table:req-RNF38}
\end{table}


\begin{table}[H]
    \centering
    \begin{tabular}{|l|p{9.5cm}|}
        \hline
        \textbf{ID} & RNF-08 \\
        \hline
        \textbf{Nombre} & Mantenimiento \\
        \hline
        \textbf{Descripción} & Se deberá tener una arquitectura bien estructurada que facilite poder añadir nuevas funcionalidades. Además, de poder corregir errores sin afectar al funcionamiento de todo el sistema  \\
        \hline
        \textbf{Prioridad} & Alta \\
        \hline
        \textbf{Estado} & Completado \\
        \hline
    \end{tabular}
    \caption{Requisito no funcional RNF-08: Mantenimiento.}
    \label{table:req-RNF39}
\end{table}

\section{Requisitos de Información}

Dentro del marco de requisitos esenciales para el desarrollo de la plataforma, los requisitos de información también tienen su papel. Estos requisitos deben definir cómo debe ser recolectada, almacenada, procesada y protegida la información dentro del sistema. 

La organización de los contenidos de los requisitos de información es la siguiente:

\begin{itemize}
    \item \textbf{ID}: Identificador único para cada requisito. Esencial para una fácil referencia y seguimiento durante todo el ciclo de vida del desarrollo del software.
    \item \textbf{Nombre}: El nombre del requisito, que se corresponde directamente con cada entidad de la base de datos para facilitar la correlación.
    \item \textbf{Descripción}: Una explicación del tipo de datos dentro del sistema.
    \item \textbf{Contenido}: Detalles específicos sobre la información que se almacena, incluyendo formatos, estructuras y cualquier restricción relevante.
\end{itemize}

\begin{table}[H]
    \centering
    \begin{tabular}{|l|p{9.5cm}|}
        \hline
        \textbf{ID} & RI-01 \\
        \hline
        \textbf{Nombre} & Usuario \\
        \hline
        \textbf{Descripción} & Datos esenciales de los usuarios para permitir el acceso y la interacción con el sistema. \\
        \hline
        \textbf{Contenido} & Información básica del usuario como nombre, apellidos, fecha de nacimiento, contacto y credenciales de acceso. También se incluye información para el seguimiento del progreso educativo y la interacción con el sistema. \\
        \hline
    \end{tabular}
    \caption{Requisito de información RI-01: Usuario.}
    \label{table:req-RI01}
\end{table}

\begin{table}[H]
    \centering
    \begin{tabular}{|l|p{9.5cm}|}
        \hline
        \textbf{ID} & RI-02 \\
        \hline
        \textbf{Nombre} & Teoría \\
        \hline
        \textbf{Descripción} &  El sistema almacenará información educativa asociada con los distintos módulos de aprendizaje. \\
        \hline
        \textbf{Contenido} & Incluye el contenido textual de la teoría, identificadores de los módulos relacionados y rutas a recursos gráficos. Se debe mantener la integridad referencial con los módulos del curso.  \\
        \hline
    \end{tabular}
    \caption{Requisito de información RI-02: Teoría.}
    \label{table:req-RI04}
\end{table}

\begin{table}[H]
    \centering
    \begin{tabular}{|l|p{9.5cm}|}
        \hline
        \textbf{ID} & RI-03 \\
        \hline
        \textbf{Nombre} & Ejercicios \\
        \hline
        \textbf{Descripción} & Conjunto de datos que definen los ejercicios prácticos dentro de los módulos del sistema.  \\
        \hline
        \textbf{Contenido} & Información detallada de los ejercicios, como el nombre, contenido descriptivo, soluciones y parámetros de evaluación, así como la identificación de si requieren revisión manual o son ejercicios clave dentro de los módulos.  \\
        \hline
    \end{tabular}
    \caption{Requisito de información RI-03: Ejercicios.}
    \label{table:req-RI05}
\end{table}

\begin{table}[H]
    \centering
    \begin{tabular}{|l|p{9.5cm}|}
        \hline
        \textbf{ID} & RI-04 \\
        \hline
        \textbf{Nombre} & Juegos \\
        \hline
        \textbf{Descripción} & Datos relacionados con los juegos disponibles en la plataforma. \\
        \hline
        \textbf{Contenido} & Título de cada juego y su costo asociado, que puede utilizarse para transacciones o recompensas dentro del sistema educativo de la plataforma. \\
        \hline
    \end{tabular}
    \caption{Requisito de información RI-04: Juegos.}
    \label{table:req-RI06}
\end{table}

\begin{table}[H]
    \centering
    \begin{tabular}{|l|p{9.5cm}|}
        \hline
        \textbf{ID} & RI-05 \\
        \hline
        \textbf{Nombre} & Progreso del estudiante \\
        \hline
        \textbf{Descripción} & Registro detallado del avance de cada estudiante en los ejercicios y módulos del curso. \\
        \hline
        \textbf{Contenido} & Información de progreso que incluye el estado del ejercicio (completado, en progreso, etc.), calificaciones, fechas relevantes, código de solución proporcionado por el estudiante, y cualquier comentario del instructor o puntuación de evaluación automática. \\
        \hline
    \end{tabular}
    \caption{Requisito de información RI-05: Progreso del estudiante.}
    \label{table:req-RI07}
\end{table}

\begin{table}[H]
    \centering
    \begin{tabular}{|l|p{9.5cm}|}
        \hline
        \textbf{ID} & RI-06 \\
        \hline
        \textbf{Nombre} & Actividad del estudiante \\
        \hline
        \textbf{Descripción} &  Información que captura las interacciones específicas de los estudiantes con los contenidos del curso. \\
        \hline
        \textbf{Contenido} & Datos sobre las acciones realizadas por el estudiante dentro de la plataforma, incluyendo si han completado o saltado actividades y el tipo de contenido con el que interactuaron, para proporcionar una visión integral de su compromiso con el material del curso. \\
        \hline
    \end{tabular}
    \caption{Requisito de información RI-06: Actividad del estudiante.}
    \label{table:req-RI08}
\end{table}

\begin{table}[H]
    \centering
    \begin{tabular}{|l|p{9.5cm}|}
        \hline
        \textbf{ID} & RI-07 \\
        \hline
        \textbf{Nombre} & Transacciones de Pagos \\
        \hline
        \textbf{Descripción} & Registros de pagos realizados por los usuarios, que detallan la adquisición de los juegos.  \\
        \hline
        \textbf{Contenido} & Detalles de cada transacción, incluyendo el usuario que realiza el pago, el juego o servicio adquirido, la cantidad pagada y la fecha del pago. \\
        \hline
    \end{tabular}
    \caption{Requisito de información RI-07: Transacciones de Pagos.}
    \label{table:req-RI09}
\end{table}

\begin{table}[H]
    \centering
    \begin{tabular}{|l|p{9.5cm}|}
        \hline
        \textbf{ID} & RI-08 \\
        \hline
        \textbf{Nombre} & Notificaciones \\
        \hline
        \textbf{Descripción} & Captura la información de las notificaciones generadas por el sistema o los profesores para los usuarios. \\
        \hline
        \textbf{Contenido} & Datos de las notificaciones como el destinatario, el mensaje específico enviado, la fecha y hora de envío, y un indicador de si la notificación ha sido leída. \\
        \hline
    \end{tabular}
    \caption{Requisito de información RI-08: Notificaciones.}
    \label{table:req-RI10}
\end{table}

\begin{table}[H]
    \centering
    \begin{tabular}{|l|p{9.5cm}|}
        \hline
        \textbf{ID} & RI-09 \\
        \hline
        \textbf{Nombre} & Requisitos \\
        \hline
        \textbf{Descripción} & Información relativa a los requisitos educativos que deben ser cumplidos por los estudiantes en el transcurso de su aprendizaje. Estos requisitos son elementos fundamentales que estructuran los módulos y el contenido del curso.  \\
        \hline
        \textbf{Contenido} & Datos que incluyen un identificador único y el nombre del requisito, que sirve como referencia para la progresión en los módulos y la vinculación con ejercicios y teoría. \\
        \hline
    \end{tabular}
    \caption{Requisito de información RI-09: Requisitos.}
    \label{table:req-RI11}
\end{table}

\begin{table}[H]
    \centering
    \begin{tabular}{|l|p{9.5cm}|}
        \hline
        \textbf{ID} & RI-10 \\
        \hline
        \textbf{Nombre} & \textit{Path} educativo \\
        \hline
        \textbf{Descripción} & Detalles que establecen la secuencia en la que los estudiantes deben cumplir con los requisitos dentro de un módulo específico. Esta secuencia asegura que el proceso de aprendizaje siga una progresión lógica y estructurada.  \\
        \hline
        \textbf{Contenido} & Datos que incluyen el identificador del módulo, el identificador del requisito y la posición que ocupa este requisito en la secuencia de aprendizaje del módulo. \\
        \hline
    \end{tabular}
    \caption{Requisito de información RI-10: \textit{Path} educativo.}
    \label{table:req-RI12}
\end{table}


\begin{table}[H]
    \centering
    \begin{tabular}{|l|p{9.5cm}|}
        \hline
        \textbf{ID} & RI-11 \\
        \hline
        \textbf{Nombre} & Módulo \\
        \hline
        \textbf{Descripción} & Información relacionada con los módulos que componen la estructura curricular de la plataforma. Cada módulo agrupa una serie de requisitos, que a su vez agrupan ejercicios y teoría para facilitar el aprendizaje en temas específicos de la programación. \\
        \hline
        \textbf{Contenido} & Datos que definen cada módulo, incluyendo un identificador único, el nombre del módulo y una descripción que detalla los objetivos y el contenido cubierto.  \\
        \hline
    \end{tabular}
    \caption{Requisito de información RI-11: Módulo.}
    \label{table:req-RI13}
\end{table}

\begin{table}[H]
    \centering
    \begin{tabular}{|l|p{9.5cm}|}
        \hline
        \textbf{ID} & RI-12 \\
        \hline
        \textbf{Nombre} & Preguntas \\
        \hline
        \textbf{Descripción} & Registra las interacciones de los estudiantes en forma de preguntas y respuestas dentro de la plataforma, lo que permite un seguimiento de las dudas y facilita la asistencia por parte del cuerpo docente o de otros estudiantes.  \\
        \hline
        \textbf{Contenido} & Incluye la identificación del estudiante que realiza la pregunta, el texto de la pregunta, la respuesta proporcionada, fechas de realización y respuesta, y cualquier archivo adjunto relevante para la discusión. \\
        \hline
    \end{tabular}
    \caption{Requisito de información RI-12: Preguntas.}
    \label{table:req-RI14}
\end{table}

\begin{table}[H]
    \centering
    \begin{tabular}{|l|p{9.5cm}|}
        \hline
        \textbf{ID} & RI-13 \\
        \hline
        \textbf{Nombre} & Requisitos completados por el usuario \\
        \hline
        \textbf{Descripción} & Contabiliza los requisitos que cada usuario ha completado en los distintos módulos del sistema. Esto permite al sistema y a los instructores monitorear el progreso y asegurar que los estudiantes cumplan con los objetivos educativos establecidos.  \\
        \hline
        \textbf{Contenido} & Incluye el identificador del usuario, el requisito completado, el módulo asociado y la fecha de finalización, permitiendo así una gestión efectiva del avance de cada estudiante. \\
        \hline
    \end{tabular}
    \caption{Requisito de información RI-13: Requisitos completados por el usuario.}
    \label{table:req-RI15}
\end{table}

\begin{table}[H]
    \centering
    \begin{tabular}{|l|p{9.5cm}|}
        \hline
        \textbf{ID} & RI-14 \\
        \hline
        \textbf{Nombre} & Requisitos de teoría \\
        \hline
        \textbf{Descripción} & Asocia los contenidos teóricos con los requisitos educativos específicos, estableciendo las dependencias que los estudiantes deben cumplir para progresar en el aprendizaje teórico del curso.  \\
        \hline
        \textbf{Contenido} & Relaciones entre teorías y requisitos que deben ser completados para acceder a dichos contenidos, asegurando una secuencia lógica de aprendizaje. \\
        \hline
    \end{tabular}
    \caption{Requisito de información RI-14: Requisitos de teoría.}
    \label{table:req-RI16}
\end{table}

\begin{table}[H]
    \centering
    \begin{tabular}{|l|p{9.5cm}|}
        \hline
        \textbf{ID} & RI-15 \\
        \hline
        \textbf{Nombre} & Requisitos de ejercicios\\
        \hline
        \textbf{Descripción} & Define las dependencias entre los ejercicios prácticos y los requisitos educativos, indicando qué conocimientos previos son necesarios para abordar cada ejercicio.  \\
        \hline
        \textbf{Contenido} & Mapeo de los ejercicios con los requisitos correspondientes, lo que permite al sistema validar la preparación del estudiante antes de intentar resolver un ejercicio. \\
        \hline
    \end{tabular}
    \caption{Requisito de información RI-15: Requisitos de los ejercicios.}
    \label{table:req-RI17}
\end{table}